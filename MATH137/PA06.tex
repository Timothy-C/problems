\input{../header}
\title{MATH 137 Fall 2020: Practice Assignment 6}

\begin{document}
\thispagestyle{firstpage}

\textbf{\@title}

\question For $f(x)=\dfrac{x+1}{x-1}$, find $f'(x)$ using the limit definition.
\begin{proof}[Solution]
  Apply the Newton quotient:
  \begin{align*}
    f'(x) & = \dlim{x}{a} \frac{f(x)-f(a)}{x-a}                           \\
          & = \dlim{x}{a} \frac{\frac{x+1}{x-1} - \frac{a+1}{a-1}}{x-a}   \\
          & = \dlim{x}{a} \frac{(x+1)(a-1) - (a+1)(x-1)}{(x-a)(x-1)(a-1)} \\
          & = \dlim{x}{a} \frac{(xa+a-x-1) - (xa-a+x-1)}{(x-a)(x-1)(a-1)} \\
          & = \dlim{x}{a} \frac{-2(x-a)}{(x-a)(x-1)(a-1)}                 \\
          & = \dlim{x}{a} \frac{-2}{(x-1)(a-1)}                           \\
          & = -\frac{2}{(x-1)^2} \qedhere
  \end{align*}
\end{proof}


\question Let $f(x)=\dfrac{ax+b}{ax-b}$ where $a \neq 0$, $b \neq 0$.
\begin{enumerate}[(a)]
  \item Find $f'(x)$ using any method.
        \begin{proof}[Solution]
          First, notice that $f(x)$ is undefined at $x=\frac{b}{a}$,
          so we differentiate along all $x \neq \frac{b}{a}$.
          Apply the quotient and linear function rules:
          \begin{align*}
            \dv{x}(\frac{ax+b}{ax-b})
             & = \frac{(ax-b)\dv{x}(ax+b) - (ax+b)\dv{x}(ax-b)}{(ax-b)^2} \\
             & = \frac{(ax-b)a - (ax+b)a}{(ax-b)^2}                       \\
             & = \frac{a(-2b)}{(ax-b)^2}                                  \\
             & = -\frac{2ab}{(ax-b)^2} \qedhere
          \end{align*}
        \end{proof}
  \item Show that for $x \neq \frac{b}{a}$, $abf'(x)<0$.
        \begin{proof}
          Let $a$ and $b$ be non-zero reals, and let $x \neq \frac{b}{a}$. Then,
          \begin{align*}
            abf'(x) & = ab\left(\frac{2ab}{(ax-b)^2}\right) \\
                    & = -\frac{2a^2b^2}{(ax-b)^2}
          \end{align*}
          Recall that the square of any non-zero number is positive.
          Then, we have that $a^2 > 0$, $b^2 > 0$, and $(ax-b)^2 > 0$.
          The last one also implies $\frac{1}{(ax-b)^2} > 0$. Multiplying,
          \begin{align*}
            \frac{a^2b^2}{(ax-b)}   & > 0          \\
            -2\frac{a^2b^2}{(ax-b)} & < 0          \\
            abf'(x)                 & < 0 \qedhere
          \end{align*}
        \end{proof}
\end{enumerate}


\question In each case, find $f'(x)$ using any method.
\begin{enumerate}[(a)]
  \item $f(x)=5^x\sin x+(x^3+x^2)\cos x$.
        \begin{proof}[Solution]
          Apply arithmetic rules and recall that $\dv{x}a^x=a^x\ln a$:
          \begin{align*}
            f'(x) & = \dv{x}(5^x\sin x)+\dv{x}((x^3+x^2)\cos x)                                         \\
                  & = (5^x\dv{x}\cos x + \cos x\dv{x}5^x)+(\cos x\dv{x}(x^3+x^2)+(x^3+x^2)\dv{x}\cos x) \\
                  & = 5^x\sin x + \ln 5\cos x5^x + \cos x(3x^2+2x)+(x^3+x^2)\sin x                      \\
                  & = \sin x(x^3+x^2+5^x) + \cos x(5^x\ln 5 + 3x^2+2x) \qedhere
          \end{align*}
        \end{proof}
  \item $f(x)=\dfrac{x^2+x-2}{x^3+6}$.
        \begin{proof}[Solution]
          Apply the quotient rule, excepting $x=\sqrt[3]{-6}$ from the domain:
          \begin{align*}
            f'(x) & = \frac{(x^3+6)\dv{x}(x^2+x-2)-(x^2+x-2)\dv{x}(x^3+6)}{(x^3+6)^2} \\
                  & = \frac{(x^3+6)(2x+1)-(x^2+x-2)3x^2}{(x^3+6)^2}                   \\
                  & = -\frac{x^4+2x^3-6x^2-12x-6}{(x^3+6)^2} \qedhere
          \end{align*}
        \end{proof}
  \item $f(x)=\sqrt{2\tan^2x+3}$.
        \begin{proof}[Solution]
          Apply the chain rule, recalling that $\dv{x}\tan x=\sec^2 x$.
          \begin{align*}
            f'(x)
             & = \dv{\sqrt{2\tan^2x+3}}{(2\tan^2x+3)}\cdot \dv{x}(2\tan^2x+3)                              \\
             & = \frac{1}{2\sqrt{2\tan^2x+3}} \cdot 2\dv{\tan^2x}{\tan x} \cdot \dv{x}\tan x               \\
             & = \frac{1}{2\sqrt{2\tan^2x+3}} \cdot \left(2\dv{\tan^2x}{\tan x} \cdot \dv{x}\tan x \right) \\
             & = \frac{1}{2\sqrt{2\tan^2x+3}} \cdot 4\tan x \sec^2x                                        \\
             & = \frac{2\tan x \sec^2x}{\sqrt{2\tan^2x+3}} \qedhere
          \end{align*}
        \end{proof}
  \item $f(x)=2^{\sin(\sec x)}$.
        \begin{proof}[Solution]
          Again, simply apply the chain rule repeatedly.
          \begin{align*}
            f'(x)
             & = \dv{(2^{\sin(\sec x)})}{(\sin(\sec x))}\cdot\dv{\sin(\sec x)}{(\sec x)}\cdot\dv{x}\sec x \\
             & = 2^{\sin(\sec x)}\ln(2)\cos(\sec x)\sec(x)\tan(x) \qedhere
          \end{align*}
        \end{proof}
\end{enumerate}


\question In each case, determine the equation of the tangent to $y=f(x)$ at the point where $x=a$.
\begin{enumerate}[(a)]
  \item $f(x)=x^2$, $a=3$.
        \begin{proof}[Solution]
          By the power rule, $f'(x) = 2x$.
          Recall the formula for the equation of a tangent: $L_a^f(x)=f(a)+f'(a)(x-a)$.
          Apply it:
          \begin{align*}
            L_3^f(x) & = f(3)+f'(3)(x-3) \\
                     & = 3^2 + 2(3)(x-3) \\
            y        & = 6x - 9 \qedhere
          \end{align*}
        \end{proof}
  \item $f(x)=\cos x$, $a=-\dfrac{3\pi}{4}$.
        \begin{proof}[Solution]
          Again, apply the linear approximation formula, knowing $f'(x)=-\sin x$.
          \begin{align*}
            L_{-3\pi/4}^f
              & = f(-\flatfrac{3\pi}{4}) + f'(-\flatfrac{3\pi}{4})\left(x-\frac{3\pi}{4}\right)      \\
              & = \cos(-\flatfrac{3\pi}{4}) - \sin(-\flatfrac{3\pi}{4})\left(x-\frac{3\pi}{4}\right) \\
              & = -\frac{\sqrt{2}}{2} + \frac{\sqrt{2}}{2}\left(x-\frac{3\pi}{4}\right)              \\
            y & = \frac{\sqrt{2}}{2}x + \frac{3\pi-2\sqrt{2}}{4} \qedhere
          \end{align*}
        \end{proof}
  \item $f(x)=e^x$, $a=\ln\pi$.
        \begin{proof}[Solution]
          Nothing new or fancy here. Even less so since $f'(x)=e^x$.
          \begin{align*}
            L_{\ln\pi}^f & = f(\ln\pi) + f'(\ln\pi)(x-\ln\pi)  \\
                         & = e^{\ln\pi} + e^{\ln\pi}(x-\ln\pi) \\
                         & = \pi + \pi(x-\ln\pi)               \\
            y            & = \pi x - \ln\pi - \pi \qedhere
          \end{align*}
        \end{proof}
  \item $f(x)=4^x$, $a=-3$.
        \begin{proof}[Solution]
          Recall the derivative of an exponential: $\dv{x}a^x = a^x\ln a$.
          \begin{align*}
            L_{-3}^f & = f(-3) + f'(-3)(x+3)                              \\
                     & = 4^{-3} + 4^{-3}\ln 4(x+3)                        \\
                     & = \frac{\ln 2}{32}x + \frac{3\ln 4+1}{64} \qedhere
          \end{align*}
        \end{proof}
\end{enumerate}


\question Compute $\displaystyle\dv{y}{x}$ and $\displaystyle\dv[2]{y}{x}$ in each case.
\begin{enumerate}[(a)]
  \item $y=\cos x^2$.
        \begin{proof}[Solution]
          Apply the chain rule to find the first derivative:
          \[ \dv{y}{x} = \dv{\cos(x^2)}{(x^2)} \cdot \dv{x^2}{x} = -2x\sin x^2. \]
          Apply the product and chain rule to find the second derivative:
          \begin{align*}
            \dv[2]{y}{x} & = \dv{x}(-2x\sin x^2)                                \\
                         & = -2\dv{x}(x\sin x^2)                                \\
                         & = -2\left( x\dv{x}\sin x^2 + \sin x^2\dv{x}x \right) \\
                         & = -2\left( x(2x\cos x^2) + \sin x^2(1) \right)       \\
                         & = -4x^2\cos x^2 - 2\sin x^2. \qedhere
          \end{align*}
        \end{proof}
  \item $y=\cos^2 x$.
        \begin{proof}[Solution]
          Follow the same procedure as above, remembering that $\cos^2 x = (\cos x)^2$:
          \[ \dv{y}{x} = \dv{\cos^2 x}{\cos x}\cdot\dv{\cos x}{x} = -2\cos x\sin x. \]
          We can simplify this to $-\sin 2x$ using the double angle identity. Then,
          \begin{align*}
            \dv[2]{y}{x} & = \dv{x}(-\sin 2x)                      \\
                         & = -\dv{\sin(2x)}{(2x)}\cdot\dv{(2x)}{x} \\
                         & = -2\cos 2x. \qedhere
          \end{align*}
        \end{proof}
\end{enumerate}


\question \begin{enumerate}[(a)]
  \item Use the Chain Rule to prove that the derivative of an even function is odd.
        \begin{proof}
          Recall that an even function $f$ is one where $f(-x)=f(x)$ for all $x$,
          and an odd function $g$ is one where $g(-x)=-g(x)$ for all $x$.
          Let $f$ be even. We must show that $f'(-x) = -f'(x)$.
          
          Notice that $(f(-x))' = f'(-x)\cdot(-x)' = -f'(-x)$.
          However, because $f$ is even, this is equal to $(f(x))' = f'(x)$.
          That is, $-f'(-x) = f'(x)$ and it follows that $f'$ is odd.
        \end{proof}
  \item Using ONLY the Chain Rule and the Product Rule (and not the Reciprocal/Quotient rules),
        give an alternative proof of the Quotient Rule.
          [Hint: $\dfrac{f(x)}{g(x)}=f(x)(g(x))^{-1}$].
        \begin{proof}
          Let $f$ and $g$ be differentiable functions, and let $h(x)=\frac{f(x)}{g(x)}$. 
          According to the above hint, write $h$ as $f(x) \cdot (g(x))^{-1}$.

          Now, apply the product rule:
          \[ h'(x) = f(x)(g(x)^{-1})' + f'(x)(g(x))^{-1} \]
          We can evaluate the derivative of $g(x)^{-1}$ using the power and chain rules:
          \[ (g(x)^{-1})' = (-1)g(x)^{-2}\cdot g'(x) = -\frac{g'(x)}{g(x)^2} \]
          Substiting back in and simplifying, we arrive at the quotient rule:
          \[ -\frac{f(x)g'(x)}{g(x)^2} + \frac{f'(x)}{g(x)} = \frac{f'(x)g(x)-f(x)g'(x)}{g(x)^2} \qedhere \]
        \end{proof}
\end{enumerate}


\question If $y=f(u)$ and $u=g(x)$ where $f$ and $g$ are twice differentiable functions, prove that
\[ \dv[2]{y}{x} = \dv[2]{y}{u}\left(\dv{u}{x}\right)^2 + \dv{y}{u}\dv[2]{u}{x}. \]
\begin{proof}
  Let $y$ be dependent on $u$ and $u$ be dependent on $x$.
  Then, by the chain rule,
  \[ \dv{y}{x} = \dv{y}{u} \cdot \dv{u}{x}. \]
  And, differentiating both sides, we have
  \[ \dv[2]{y}{x} = \dv{x}(\dv{y}{u} \cdot \dv{u}{x}) \]
  which is just a product, so we may apply the product rule. This gives
  \begin{align*}
    \dv[2]{y}{x} & = \dv{x}(\dv{y}{u})\cdot\dv{u}{x} + \dv{y}{u}\cdot\dv{x}(\dv{u}{x})  \\
                 & = \dv{y}{u}\cdot\dv{u}{x}\cdot\dv{u}{x} + \dv{y}{u}\cdot\dv[2]{u}{x} \\
                 & = \dv[2]{y}{u}\left(\dv{u}{x}\right)^2 + \dv{y}{u}\dv[2]{u}{x}
  \end{align*}
  exactly as desired.
\end{proof}

\end{document}